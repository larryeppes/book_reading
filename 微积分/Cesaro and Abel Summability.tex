%% LyX 2.3.6.1 created this file.  For more info, see http://www.lyx.org/.
%% Do not edit unless you really know what you are doing.
\documentclass[a4paper,UTF8]{article}
\usepackage[T1]{fontenc}
\usepackage{CJKutf8}
\usepackage{geometry}
\geometry{verbose,tmargin=2cm,bmargin=2cm,lmargin=2cm,rmargin=2cm}
\usepackage{color}
\usepackage{mathtools}
\usepackage{amsmath}
\usepackage{amsthm}
\usepackage{amssymb}
\usepackage{esint}
\usepackage[unicode=true,pdfusetitle,
 bookmarks=true,bookmarksnumbered=false,bookmarksopen=false,
 breaklinks=false,pdfborder={0 0 0},pdfborderstyle={},backref=false,colorlinks=true]
 {hyperref}

\makeatletter
%%%%%%%%%%%%%%%%%%%%%%%%%%%%%% Textclass specific LaTeX commands.
\theoremstyle{plain}
    \ifx\thechapter\undefined
	    \newtheorem{thm}{\protect\theoremname}
	  \else
      \newtheorem{thm}{\protect\theoremname}[chapter]
    \fi

%%%%%%%%%%%%%%%%%%%%%%%%%%%%%% User specified LaTeX commands.
% 如果没有这一句命令,XeTeX会出错,原因参见
% http://bbs.ctex.org/viewthread.php?tid=60547
\DeclareRobustCommand\nobreakspace{\leavevmode\nobreak\ }
% \usepackage{tkz-euclide}
% \usetkzobj{all}

\usepackage{amsmath, amsthm, amsfonts, amssymb, mathtools, yhmath, mathrsfs}
% http://ctan.org/pkg/extarrows
% long equal sign
\usepackage{extarrows}

\DeclareMathOperator{\sech}{sech}

%\everymath{\color{blue}\everymath{}}
\everymath\expandafter{\color{blue}\displaystyle}
%\everydisplay\expandafter{\the\everydisplay \color{red}}

\def\degree{^\circ}
\def\bt{\begin{theorem}}
\def\et{\end{theorem}}
\def\bl{\begin{lemma}}
\def\el{\end{lemma}}
\def\bc{\begin{corrolary}}
\def\ec{\end{corrolary}}
\def\ba{\begin{proof}[解]}
\def\ea{\end{proof}}
\def\ue{\mathrm{e}}
\def\ud{\,\mathrm{d}}
\def\GF{\mathrm{GF}}
\def\ui{\mathrm{i}}
\def\Re{\mathrm{Re}}
\def\Im{\mathrm{Im}}
\def\uRes{\mathrm{Res}}
\def\diag{\,\mathrm{diag}\,}
\def\be{\begin{equation}}
\def\ee{\end{equation}}
\def\bee{\begin{equation*}}
\def\eee{\end{equation*}}
\def\sumcyc{\sum\limits_{cyc}}
\def\prodcyc{\prod\limits_{cyc}}
\def\i{\infty}
\def\a{\alpha}
\def\b{\beta}
\def\g{\gamma}
\def\d{\delta}
\def\l{\lambda}
\def\m{\mu}
\def\t{\theta}
\def\p{\partial}
\def\wc{\rightharpoonup}
\def\udiv{\mathrm{div}}
\def\diam{\mathrm{diam}}
\def\dist{\mathrm{dist}}
\def\uloc{\mathrm{loc}}
\def\uLip{\mathrm{Lip}}
\def\ucurl{\mathrm{curl}}
\def\usupp{\mathrm{supp}}
\def\uspt{\mathrm{spt}}

%%%%%%%%%%%%%%%%%%%%%%%%%%%%%%%%%%%%%%%%%%%%%%%%%%%%%%%%%%%%%%%%%%%%%%%%%%%%%%%%%%%
%%%%%%%%%%%%%%%%%%%%%%%%%%%%%%%%%%%%%%%%%%%%%%%%%%%%%%%%%%%%%%%%%%%%%%%%%%%%%%%%%%%
%%%%%%%%%%%%%%%%%%%%%%%%%%%%%%%%%%%%%%%%%%%%%%%%%%%%%%%%%%%%%%%%%%%%%%%%%%%%%%%%%%%

\providecommand{\abs}[1]{\left\lvert#1\right\rvert}
\providecommand{\norm}[1]{\left\Vert#1\right\Vert}
\providecommand{\paren}[1]{\left(#1\right)}

%%%%%%%%%%%%%%%%%%%%%%%%%%%%%%%%%%%%%%%%%%%%%%%%%%%%%%%%%%%%%%%%%%%%%%%%%%%%%%%%%%%
%%%%%%%%%%%%%%%%%%%%%%%%%%%%%%%%%%%%%%%%%%%%%%%%%%%%%%%%%%%%%%%%%%%%%%%%%%%%%%%%%%%
%%%%%%%%%%%%%%%%%%%%%%%%%%%%%%%%%%%%%%%%%%%%%%%%%%%%%%%%%%%%%%%%%%%%%%%%%%%%%%%%%%%

\newcommand{\FF}{\mathbb{F}}
\newcommand{\ZZ}{\mathbb{Z}}
\newcommand{\WW}{\mathbb{W}}
\newcommand{\NN}{\mathbb{N}}
\newcommand{\PP}{\mathbb{P}}
\newcommand{\QQ}{\mathbb{Q}}
\newcommand{\RR}{\mathbb{R}}
\newcommand{\TT}{\mathbb{T}}
\newcommand{\CC}{\mathbb{C}}
\newcommand{\pNN}{\mathbb{N}_{+}}
\newcommand{\cZ}{\mathcal{Z}}
\newcommand{\cS}{\mathcal{S}}
\newcommand{\cX}{\mathcal{X}}
\newcommand{\cW}{\mathcal{W}}

\newcommand{\eqdef}{\xlongequal{\text{def}}}%
\newcommand{\eqexdef}{\xlongequal[\text{存在}]{\text{记为}}}%

\makeatother

\providecommand{\theoremname}{定理}

\begin{document}
\begin{CJK}{UTF8}{gbsn}%
\title{Ces\`{a}ro and Abel Summability}

\maketitle
考虑序列$\left\{ a_{n}\right\} _{n\ge0}$, 它的部分和记为
\[
s_{n}=a_{0}+\cdots+a_{n}.
\]
有时, 序列$\left\{ s_{n}\right\} $不收敛, 所以这里从另一个角度来研究部分和. 注意当$a_{n}$收敛时,
$\frac{s_{n}}{n}$也收敛, 并与$a_{n}$的极限一致, 不要将此性质与序列的Ces\`{a}ro和混淆,
注意区分$\frac{s_{n}}{n}$和$\sigma_{n}$表达式上的不同. 定义$\left\{ a_{n}\right\} $的Ces\`{a}ro和如下
\[
\sigma_{n}=\frac{1}{n}(s_{0}+\cdots+s_{n-1}).
\]
这和对于不收敛的序列$\left\{ a_{n}\right\} $可能是收敛的. 对于任意的$r\in[0,1)$, 可以考虑加权得到Abel和
\[
A(r)=\sum_{k=0}^{+\infty}a_{k}r^{k},
\]
当$\lim_{r\to1}A(r)$存在且有限时, 称这个极限为序列$\left\{ a_{n}\right\} $的Abel极限.
Abel求和公式变换上式有
\begin{align*}
\sum_{k=0}^{N}a_{k}r^{k} & =(1-r)\sum_{k=0}^{N}s_{k}r^{k}+s_{N}r^{N+1}.
\end{align*}
当$N\to+\infty$, 有$s_{N}r^{N+1}\to0$时. Abel和不是别的, 就是序列$s_{k}$的带权$(1-r)r^{k}$的和.

\paragraph{性质: }

部分和序列收敛蕴含Ces\`{a}ro和收敛, Ces\`{a}ro和收敛蕴含Abel和收敛.

\paragraph{证明:}

1. 证明主要是对下面的和式拆分进行描述说明, 其中$s_{n}\to s$, $n\to+\infty$.
\[
\sigma_{n}-s=\frac{1}{n}\sum_{k=0}^{N-1}(s_{k}-s)+\frac{1}{n}\sum_{k=N}^{n-1}(s_{k}-s).
\]

2. 当$r<1$时, 容易得到
\[
\sum_{k=0}^{+\infty}a_{k}r^{k}=(1-r)^{2}\sum_{k=0}^{+\infty}k\sigma_{k}r^{k-1},\qquad r\in(0,1).
\]
设$\sigma_{n}\to s$, $n\to+\infty$. 注意下式的分解
\[
(1-r)^{2}\sum_{k=0}^{+\infty}k\sigma_{k}r^{k-1}-s=(1-r)^{2}\sum_{k=1}^{N-1}kr^{k-1}\left(\sigma_{k}-s\right)+(1-r)^{2}\sum_{k=N}^{+\infty}kr^{k-1}\left(\sigma_{k}-s\right).
\]


\paragraph{序列不收敛, 但Ces\`{a}ro和收敛的例子
\[
1-1+1-1+1-1+\cdots.
\]
}

\paragraph{Ces\`{a}ro和不收敛但Abel和收敛的例子}

\[
1-2+3-4+5-6+\cdots.
\]

1897年, Tauber给出了Abel和收敛可以推出原序列收敛的一个充分条件. 
\begin{thm}
(Tauber, 1897). 设序列$\left\{ a_{k}\right\} $为复数列, 假设
\[
A(r)=\sum_{k=0}^{+\infty}a_{k}r^{k}
\]
对于任意的$r\in(0,1)$收敛. 若$A(r)\to A$, $r\to1^{-}$. 且$ka_{k}=o(1)$,
则有
\[
\sum_{k=0}^{+\infty}a_{k}=A.
\]
\end{thm}

\paragraph{证明: }

因为$\sum a_{k}r^{k}\to A$, $r\to1^{-}$, 只需证
\[
\sum_{k=0}^{N}a_{k}-\sum_{k=0}^{+\infty}a_{k}r^{k}\rightarrow0
\]
沿着$r=1-\frac{1}{N}$, $N\to+\infty$成立即可. 对于任意的$\epsilon>0$, 存在$K$使得,
只要$k>K$, 就有$\left|ka_{k}\right|>\epsilon$. 于是记
\[
\begin{aligned}\sum_{k=0}^{N}a_{k}-\sum_{k=0}^{+\infty}a_{k}r^{k} & =\sum_{k=0}^{K}a_{k}\left(1-r^{k}\right)+\sum_{k=K+1}^{N}a_{k}\left(1-r^{k}\right)-\sum_{k=N+1}^{+\infty}a_{k}r^{k}\\
 & =I_{1}+I_{2}+I_{3}.
\end{aligned}
\]
显然, 当$N\to+\infty$时有$I_{1}\to0$. 对于$I_{2}$, 有
\[
\begin{aligned}\left|I_{2}\right| & \leq(1-r)\sum_{k=K+1}^{N}\left|a_{k}\right|\left(1+\cdots+r^{k-1}\right)\\
 & \leq(1-r)\sum_{k=K+1}^{N}\left|ka_{k}\right|\\
 & \leq(1-r)N\epsilon\leq\epsilon,
\end{aligned}
\]
注意上式中的$r$是沿着$1-\frac{1}{N}$趋向于$1$. 对于$I_{3}$, 有
\[
\left|I_{3}\right|\leq\sum_{k=N+1}^{+\infty}\left|ka_{k}\right|\frac{r^{k}}{k}<\frac{\epsilon}{N+1}\sum_{k=0}^{+\infty}r^{k}<\epsilon.
\]

Littlewood, 1911年将上面定理的$o(1)$弱化为$\mathcal{O}(1)$. Tauberian定理一般用来从级数的Abel可和性推出实可和性.

Remark: 对于正项级数, 以上三种收敛是互相等价的. 事实上, $s_{n}$是递增的必有极限, (极限可能是$+\infty$),
从而Ces\`{a}ro和与Abel和有相同的极限.
\end{CJK}

\end{document}
